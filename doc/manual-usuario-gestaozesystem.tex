\documentclass[12pt,a4paper]{article}
\usepackage[utf8]{inputenc}
\usepackage[portuguese]{babel}
\usepackage[T1]{fontenc}
\usepackage{amsmath}
\usepackage{amsfonts}
\usepackage{amssymb}
\usepackage{graphicx}
\usepackage{url}
\usepackage{hyperref}
\usepackage{xcolor}
\usepackage{listings}
\usepackage{geometry}
\usepackage{fancyhdr}
\usepackage{tcolorbox}
\usepackage{enumitem}
\usepackage{booktabs}
\usepackage{longtable}
\usepackage{array}
\usepackage{tikz}
\usepackage{pagecolor}
\usepackage{afterpage}


% Configurações da página
\geometry{
    a4paper,
    left=3cm,
    right=2cm,
    top=3cm,
    bottom=2cm
}

% Configurações do header/footer
\pagestyle{fancy}
\fancyhf{}
\fancyhead[L]{\leftmark}
\fancyhead[R]{\thepage}
\renewcommand{\headrulewidth}{0.4pt}

% Cores personalizadas
\definecolor{primary}{RGB}{59, 130, 246}
\definecolor{secondary}{RGB}{107, 114, 128}
\definecolor{success}{RGB}{34, 197, 94}
\definecolor{warning}{RGB}{251, 146, 60}
\definecolor{error}{RGB}{239, 68, 68}
\definecolor{codebackground}{RGB}{248, 250, 252}
\definecolor{lightblue}{RGB}{240, 248, 255}
\definecolor{restaurantblue}{RGB}{41, 98, 255}
\definecolor{restaurantgold}{RGB}{255, 193, 7}

% Configurar fundo azul claro
\pagecolor{lightblue}

% Configuração de código
\lstset{
    basicstyle=\ttfamily\small,
    backgroundcolor=\color{white},
    frame=single,
    framerule=1pt,
    frameround=tttt,
    rulecolor=\color{restaurantblue},
    breaklines=true,
    captionpos=b,
    numbers=left,
    numberstyle=\tiny\color{secondary},
    keywordstyle=\color{restaurantblue}\bfseries,
    commentstyle=\color{secondary}\itshape,
    stringstyle=\color{success},
    xleftmargin=10pt,
    xrightmargin=10pt
}

% Configuração de hyperlinks
\hypersetup{
    colorlinks=true,
    linkcolor=primary,
    filecolor=primary,
    urlcolor=primary,
    pdftitle={Manual do Usuário - GestãoZe System},
    pdfauthor={Sistema de Gestão de Estoque},
    pdfsubject={Manual Técnico e Guia de Utilização}
}

% Caixas de destaque aprimoradas
\newtcolorbox{infobox}[2][]{
    colback=white,
    colframe=restaurantblue,
    fonttitle=\bfseries,
    title=#2,
    arc=8pt,
    boxrule=2pt,
    left=10pt,
    right=10pt,
    top=8pt,
    bottom=8pt,
    drop shadow={0.5mm}{0.5mm}{0.2mm}{black!20},
    #1
}

\newtcolorbox{warningbox}[2][]{
    colback=white,
    colframe=warning,
    fonttitle=\bfseries,
    title=#2,
    arc=8pt,
    boxrule=2pt,
    left=10pt,
    right=10pt,
    top=8pt,
    bottom=8pt,
    drop shadow={0.5mm}{0.5mm}{0.2mm}{black!20},
    #1
}

\newtcolorbox{successbox}[2][]{
    colback=white,
    colframe=success,
    fonttitle=\bfseries,
    title=#2,
    arc=8pt,
    boxrule=2pt,
    left=10pt,
    right=10pt,
    top=8pt,
    bottom=8pt,
    drop shadow={0.5mm}{0.5mm}{0.2mm}{black!20},
    #1
}

% Capa personalizada
\newcommand{\makecustomtitle}{%
    \begin{titlepage}
        \pagecolor{white}
        \begin{tikzpicture}[remember picture,overlay]
            % Fundo gradient azul
            \fill[restaurantblue!20] (current page.south west) rectangle (current page.north east);

            % Header decorativo
            \fill[restaurantblue] (current page.north west) rectangle ([yshift=-3cm]current page.north east);

            % Footer decorativo
            \fill[restaurantblue] (current page.south west) rectangle ([yshift=2cm]current page.south east);

            % Elementos decorativos dourados
            \fill[restaurantgold] ([xshift=2cm,yshift=-1cm]current page.north west) circle (0.5cm);
            \fill[restaurantgold] ([xshift=-2cm,yshift=-1cm]current page.north east) circle (0.5cm);
            \fill[restaurantgold] ([xshift=2cm,yshift=1cm]current page.south west) circle (0.5cm);
            \fill[restaurantgold] ([xshift=-2cm,yshift=1cm]current page.south east) circle (0.5cm);
        \end{tikzpicture}

        \vspace*{1cm}
        \begin{center}
            % Logo placeholder (pode ser substituído por imagem real)
            \begin{tikzpicture}
                \fill[white] (0,0) circle (2cm);
                \fill[restaurantblue] (0,0) circle (1.8cm);
                \node[white, font=\Huge\bfseries] at (0,0.3) {PdC};
                \node[white, font=\large] at (0,-0.5) {RESTAURANT};
            \end{tikzpicture}

            \vspace{1cm}

            {\color{restaurantblue}\Huge\textbf{PEDACINHO DO CÉU}}

            \vspace{0.5cm}

            {\color{restaurantblue}\Large\textit{Restaurante \& Gastronomia}}

            \vspace{1.5cm}

            {\color{restaurantblue}\LARGE\textbf{GESTÃOZE SYSTEM}}

            \vspace{0.5cm}

            {\color{secondary}\Large Sistema de Gestão de Estoque Inteligente}

            \vspace{0.8cm}

            {\color{restaurantblue}\large Manual Técnico e Guia de Utilização}

            \vspace{2cm}

            % Informações técnicas
            \begin{tcolorbox}[
                colback=white,
                colframe=restaurantblue,
                width=12cm,
                arc=10pt,
                boxrule=2pt
            ]
                \centering
                {\color{restaurantblue}\large\textbf{Especificações Técnicas}}

                \vspace{0.5cm}

                \begin{tabular}{ll}
                    \textbf{Frontend:} & Vue.js 3.5.21 + TypeScript \\
                    \textbf{Backend:} & Supabase (PostgreSQL) \\
                    \textbf{UI Framework:} & Components Customizados \\
                    \textbf{Charts:} & Chart.js 4.5.0 \\
                    \textbf{AI Integration:} & Google Gemini AI \\
                    \textbf{Versão:} & 1.0.0 \\
                \end{tabular}
            \end{tcolorbox}

            \vspace{1cm}

            % Data e informações
            {\color{restaurantblue}\large Desenvolvido especialmente para}

            {\color{restaurantblue}\large\textbf{Gestão de Estoque em Restaurantes}}

            \vspace{1cm}

            {\color{secondary}\normalsize\today}

        \end{center}

        % Rodapé da capa
        \vfill
        \begin{center}
            {\color{white}\large\textbf{Sistema Completo de Gestão}}

            {\color{white}\normalsize Controle de Estoque | Análise Preditiva | Relatórios Avançados}
        \end{center}

    \end{titlepage}
    \pagecolor{lightblue}
}

\begin{document}

\makecustomtitle

\newpage
\tableofcontents
\newpage

% ========================================
% SEÇÃO 1: INTRODUÇÃO
% ========================================

\section{Introdução}

\subsection{Visão Geral do Sistema}

O \textbf{GestãoZe System} é uma aplicação web moderna de gestão de estoque desenvolvida com tecnologias de ponta para atender às necessidades de pequenas e médias empresas do setor alimentício. O sistema combina funcionalidades tradicionais de controle de inventário com recursos avançados de inteligência artificial, análise preditiva e relatórios automatizados.

\begin{infobox}{Características Principais}
\begin{itemize}[noitemsep]
    \item Interface moderna e responsiva construída em Vue.js 3
    \item Backend robusco com Supabase (PostgreSQL)
    \item Análise de dados avançada com inteligência artificial
    \item Relatórios dinâmicos e exportação em múltiplos formatos
    \item Sistema de autenticação e autorização seguro
    \item Gerenciamento completo de produtos, fornecedores e cardápios
    \item Análise preditiva de demanda e otimização de estoque
\end{itemize}
\end{infobox}

\subsection{Tecnologias Utilizadas}

\subsubsection{Frontend}
\begin{itemize}
    \item \textbf{Vue.js 3.5.21}: Framework JavaScript progressivo para construção da interface
    \item \textbf{TypeScript 5.2.2}: Superset do JavaScript com tipagem estática
    \item \textbf{Pinia 2.3.1}: Gerenciamento de estado reativo
    \item \textbf{Vue Router 4.5.1}: Roteamento SPA (Single Page Application)
    \item \textbf{Chart.js 4.5.0}: Biblioteca para gráficos e visualizações
    \item \textbf{Lucide Vue Next}: Biblioteca de ícones moderna
\end{itemize}

\subsubsection{Backend e Banco de Dados}
\begin{itemize}
    \item \textbf{Supabase}: Backend-as-a-Service com PostgreSQL
    \item \textbf{PostgreSQL}: Banco de dados relacional robusto
    \item \textbf{Row Level Security (RLS)}: Segurança em nível de linha
    \item \textbf{Real-time subscriptions}: Atualizações em tempo real
\end{itemize}

\subsubsection{Ferramentas Auxiliares}
\begin{itemize}
    \item \textbf{Vite}: Build tool e servidor de desenvolvimento
    \item \textbf{Axios}: Cliente HTTP para requisições
    \item \textbf{Date-fns}: Manipulação de datas
    \item \textbf{jsPDF}: Geração de relatórios PDF
    \item \textbf{html2canvas}: Captura de elementos HTML
    \item \textbf{XLSX}: Manipulação de planilhas Excel
\end{itemize}

\subsection{Arquitetura do Sistema}

\subsubsection{Padrão de Arquitetura}

O sistema segue uma arquitetura de \textbf{Single Page Application (SPA)} baseada em componentes, implementando os seguintes padrões:

\begin{itemize}
    \item \textbf{Model-View-ViewModel (MVVM)}: Implementado através do Vue.js
    \item \textbf{Service Layer Pattern}: Separação da lógica de negócio em serviços
    \item \textbf{Store Pattern}: Gerenciamento de estado centralizado com Pinia
    \item \textbf{Component-Based Architecture}: Interface modular e reutilizável
\end{itemize}

\subsubsection{Estrutura de Diretórios}

\begin{lstlisting}[language=bash, caption=Estrutura do Projeto]
src/
├── components/          # Componentes reutilizáveis
├── views/              # Páginas da aplicação
├── services/           # Camada de serviços
├── stores/             # Gerenciamento de estado (Pinia)
├── router/             # Configuração de rotas
├── config/             # Configurações (Supabase, etc.)
├── types/              # Definições de tipos TypeScript
├── utils/              # Funções utilitárias
└── assets/             # Recursos estáticos
\end{lstlisting}

% ========================================
% SEÇÃO 2: CONFIGURAÇÃO E INSTALAÇÃO
% ========================================

\section{Configuração e Instalação}

\subsection{Requisitos do Sistema}

\subsubsection{Requisitos de Software}
\begin{itemize}
    \item \textbf{Node.js}: Versão 18.x ou superior
    \item \textbf{NPM}: Versão 9.x ou superior
    \item \textbf{Navegador}: Chrome 90+, Firefox 88+, Safari 14+, Edge 90+
\end{itemize}

\subsubsection{Configuração do Ambiente}

\begin{warningbox}{Variáveis de Ambiente Necessárias}
Crie um arquivo \texttt{.env} na raiz do projeto com as seguintes variáveis:
\begin{lstlisting}
VITE_SUPABASE_URL=sua_url_do_supabase
VITE_SUPABASE_ANON_KEY=sua_chave_anonima_supabase
VITE_GEMINI_API_KEY=sua_chave_api_gemini
\end{lstlisting}
\end{warningbox}

\subsection{Instalação}

\begin{enumerate}
    \item \textbf{Clone o repositório}:
    \begin{lstlisting}[language=bash]
git clone <repository_url>
cd gestaozesystem-web
    \end{lstlisting}

    \item \textbf{Instale as dependências}:
    \begin{lstlisting}[language=bash]
npm install
    \end{lstlisting}

    \item \textbf{Configure as variáveis de ambiente}:
    \begin{lstlisting}[language=bash]
cp .env.example .env
# Edite o arquivo .env com suas credenciais
    \end{lstlisting}

    \item \textbf{Execute em modo de desenvolvimento}:
    \begin{lstlisting}[language=bash]
npm run dev
    \end{lstlisting}

    \item \textbf{Build para produção}:
    \begin{lstlisting}[language=bash]
npm run build
    \end{lstlisting}
\end{enumerate}

% ========================================
% SEÇÃO 3: BANCO DE DADOS E SUPABASE
% ========================================

\section{Banco de Dados e Supabase}

\subsection{Arquitetura do Banco de Dados}

O sistema utiliza PostgreSQL através do Supabase, implementando uma estrutura relacional otimizada para gestão de estoque de estabelecimentos alimentícios.

\subsubsection{Tabelas Principais}

\begin{longtable}{|p{3cm}|p{4cm}|p{8cm}|}
\hline
\textbf{Tabela} & \textbf{Propósito} & \textbf{Campos Principais} \\
\hline
\endhead

\texttt{admin\_users} & Usuários do sistema & id, email, password\_hash, role, created\_at \\
\hline

\texttt{produtos} & Produtos em estoque & id, nome, categoria, current\_stock, min\_stock, max\_stock, unit\_cost, sale\_price, supplier\_id \\
\hline

\texttt{categorias} & Categorias de produtos & id, nome, description, color\_tag \\
\hline

\texttt{movements} & Movimentações de estoque & id, product\_id, type, quantity, reason, user\_id, created\_at \\
\hline

\texttt{menu\_items} & Itens do cardápio & id, nome, category, price, description, is\_available \\
\hline

\texttt{menu\_item\_ingredientes} & Ingredientes dos pratos & id, menu\_item\_id, product\_id, quantity\_needed \\
\hline

\texttt{menu\_diario} & Cardápio diário & id, date, menu\_item\_id, quantity\_planned \\
\hline

\texttt{planejamento\_semanal} & Planejamento semanal & id, week\_start, menu\_item\_id, quantities\_by\_day \\
\hline

\texttt{suppliers} & Fornecedores & id, name, contact\_info, email, phone, address \\
\hline

\texttt{reports} & Relatórios salvos & id, name, type, data, user\_id, created\_at \\
\hline

\texttt{logs} & Logs do sistema & id, action, details, user\_id, ip\_address, created\_at \\
\hline

\texttt{app\_settings} & Configurações & key, value, description, updated\_at \\
\hline

\end{longtable}

\subsubsection{Relacionamentos e Constraints}

\begin{itemize}
    \item \textbf{produtos.supplier\_id} $\rightarrow$ \textbf{suppliers.id}: Relacionamento many-to-one
    \item \textbf{movements.product\_id} $\rightarrow$ \textbf{produtos.id}: Relacionamento many-to-one
    \item \textbf{menu\_item\_ingredientes.product\_id} $\rightarrow$ \textbf{produtos.id}: Relacionamento many-to-one
    \item \textbf{menu\_item\_ingredientes.menu\_item\_id} $\rightarrow$ \textbf{menu\_items.id}: Relacionamento many-to-one
\end{itemize}

\subsection{Configuração do Supabase}

\subsubsection{Row Level Security (RLS)}

O sistema implementa políticas de segurança em nível de linha para garantir o acesso controlado aos dados:

\begin{lstlisting}[language=sql, caption=Exemplo de Política RLS]
-- Política para tabela de produtos
CREATE POLICY "Users can view products" ON produtos
    FOR SELECT USING (auth.role() = 'authenticated');

CREATE POLICY "Admin can insert products" ON produtos
    FOR INSERT WITH CHECK (auth.jwt() ->> 'role' = 'admin');

CREATE POLICY "Admin can update products" ON produtos
    FOR UPDATE USING (auth.jwt() ->> 'role' = 'admin');
\end{lstlisting}

\subsubsection{Triggers e Funções}

\begin{lstlisting}[language=sql, caption=Trigger para Atualização de Estoque]
-- Função para atualizar estoque após movimentação
CREATE OR REPLACE FUNCTION update_product_stock()
RETURNS TRIGGER AS $$
BEGIN
    IF NEW.type = 'entrada' THEN
        UPDATE produtos
        SET current_stock = current_stock + NEW.quantity
        WHERE id = NEW.product_id;
    ELSIF NEW.type = 'saida' THEN
        UPDATE produtos
        SET current_stock = current_stock - NEW.quantity
        WHERE id = NEW.product_id;
    END IF;

    RETURN NEW;
END;
$$ LANGUAGE plpgsql;

-- Trigger
CREATE TRIGGER trigger_update_stock
    AFTER INSERT ON movements
    FOR EACH ROW
    EXECUTE FUNCTION update_product_stock();
\end{lstlisting}

% ========================================
% SEÇÃO 4: SISTEMA DE ROTAS
% ========================================

\section{Sistema de Rotas}

\subsection{Configuração de Roteamento}

O sistema utiliza Vue Router 4 com roteamento baseado em história (History Mode), proporcionando URLs limpas e navegação SPA otimizada.

\subsubsection{Estrutura de Rotas}

\begin{longtable}{|p{2.5cm}|p{3cm}|p{3cm}|p{6cm}|}
\hline
\textbf{Rota} & \textbf{Componente} & \textbf{Autenticação} & \textbf{Descrição} \\
\hline
\endhead

\texttt{/} & - & - & Redirecionamento para /dashboard \\
\hline

\texttt{/login} & LoginView & Convidado & Página de autenticação \\
\hline

\texttt{/dashboard} & DashboardView & Obrigatória & Painel principal com métricas \\
\hline

\texttt{/inventory} & InventoryView & Obrigatória & Gestão de produtos e estoque \\
\hline

\texttt{/ai} & AIView & Obrigatória & Análises com inteligência artificial \\
\hline

\texttt{/reports} & ReportsView & Obrigatória & Relatórios e análises avançadas \\
\hline

\texttt{/suppliers} & SuppliersView & Obrigatória & Gestão de fornecedores \\
\hline

\texttt{/menu} & MenuView & Obrigatória & Gestão de cardápio e planejamento \\
\hline

\texttt{/logs} & LogsView & Obrigatória & Auditoria e logs do sistema \\
\hline

\texttt{/settings} & SettingsView & Obrigatória & Configurações do sistema \\
\hline

\texttt{/profile} & ProfileView & Obrigatória & Perfil do usuário \\
\hline

\texttt{/about} & AboutView & Obrigatória & Informações sobre o sistema \\
\hline

\end{longtable}

\subsubsection{Guards de Navegação}

O sistema implementa guards de navegação para controle de acesso:

\begin{lstlisting}[language=typescript, caption=Guard de Autenticação]
router.beforeEach(async (to) => {
  const authStore = useAuthStore()

  // Verificar sessão armazenada
  const stored = localStorage.getItem('userSession')
  if (stored && !authStore.user) {
    authStore.user = JSON.parse(stored)
  }

  const isAuthenticated = authStore.isAuthenticated

  if (to.meta.requiresAuth && !isAuthenticated) {
    return '/login'
  }

  if (to.meta.requiresGuest && isAuthenticated) {
    return '/dashboard'
  }
})
\end{lstlisting}

% ========================================
% SEÇÃO 5: GUIA DE UTILIZAÇÃO - ROTAS
% ========================================

\section{Guia de Utilização das Rotas}

\subsection{Rota /login - Autenticação}

\subsubsection{Funcionalidades}
\begin{itemize}
    \item Autenticação de usuários via email e senha
    \item Validação de credenciais com Supabase Auth
    \item Persistência de sessão local
    \item Redirecionamento automático após login
\end{itemize}

\subsubsection{Uso da Rota}
\begin{enumerate}
    \item Acesse \texttt{http://localhost:5173/login}
    \item Insira credenciais válidas (email e senha)
    \item Clique em "Entrar"
    \item Sistema redireciona automaticamente para \texttt{/dashboard}
\end{enumerate}

\begin{warningbox}{Segurança}
As senhas são criptografadas pelo Supabase Auth. Nunca são armazenadas em texto plano.
\end{warningbox}

\subsubsection{Serviços Utilizados}
\begin{itemize}
    \item \texttt{authService.ts}: Gerencia autenticação
    \item \texttt{auth.ts} (store): Estado de autenticação global
\end{itemize}

\subsection{Rota /dashboard - Painel Principal}

\subsubsection{Funcionalidades}
\begin{itemize}
    \item Visão geral de métricas importantes
    \item Gráficos de vendas e movimentação
    \item Alertas de estoque baixo
    \item Resumo de produtos críticos
    \item Cards informativos com KPIs
\end{itemize}

\subsubsection{Estrutura da Página}
\begin{enumerate}
    \item \textbf{Header}: Métricas rápidas (total de produtos, valor do estoque)
    \item \textbf{Gráficos}: Visualizações de dados de vendas e movimentação
    \item \textbf{Alertas}: Produtos com estoque baixo ou zerado
    \item \textbf{Atividades Recentes}: Últimas movimentações
\end{enumerate}

\subsubsection{Serviços Utilizados}
\begin{itemize}
    \item \texttt{productService.ts}: Dados de produtos
    \item \texttt{salesService.ts}: Dados de vendas
    \item \texttt{reportsService.ts}: Métricas e análises
\end{itemize}

\subsection{Rota /inventory - Gestão de Estoque}

\subsubsection{Funcionalidades Principais}
\begin{itemize}
    \item CRUD completo de produtos
    \item Controle de movimentação (entrada/saída)
    \item Gestão de categorias
    \item Filtros e busca avançada
    \item Importação/exportação de dados
    \item Código de barras e QR Code
\end{itemize}

\subsubsection{Operações de Produto}

\paragraph{Cadastro de Produto}
\begin{enumerate}
    \item Clique em "Novo Produto"
    \item Preencha os campos obrigatórios:
    \begin{itemize}
        \item Nome do produto
        \item Categoria
        \item Estoque atual
        \item Estoque mínimo
        \item Preço de custo
        \item Preço de venda
        \item Fornecedor
    \end{itemize}
    \item Clique em "Salvar"
\end{enumerate}

\paragraph{Movimentação de Estoque}
\begin{enumerate}
    \item Selecione o produto na lista
    \item Clique em "Movimentar"
    \item Escolha o tipo (Entrada/Saída)
    \item Informe a quantidade
    \item Adicione observações (opcional)
    \item Confirme a movimentação
\end{enumerate}

\subsubsection{Filtros e Busca}
\begin{itemize}
    \item \textbf{Busca por nome}: Campo de pesquisa em tempo real
    \item \textbf{Filtro por categoria}: Dropdown com categorias disponíveis
    \item \textbf{Filtro por status}: Estoque normal, baixo, zerado
    \item \textbf{Filtro por fornecedor}: Produtos de fornecedor específico
\end{itemize}

\subsubsection{Serviços Utilizados}
\begin{itemize}
    \item \texttt{productService.ts}: Operações CRUD de produtos
    \item \texttt{movementService.ts}: Controle de movimentações
    \item \texttt{categoryService.ts}: Gestão de categorias
\end{itemize}

\subsection{Rota /suppliers - Gestão de Fornecedores}

\subsubsection{Funcionalidades}
\begin{itemize}
    \item Cadastro completo de fornecedores
    \item Controle de contatos e informações comerciais
    \item Histórico de compras por fornecedor
    \item Avaliação de performance
    \item Gestão de contratos e prazos
\end{itemize}

\subsubsection{Cadastro de Fornecedor}
\begin{enumerate}
    \item Clique em "Novo Fornecedor"
    \item Preencha as informações:
    \begin{itemize}
        \item Razão social
        \item Nome fantasia
        \item CNPJ/CPF
        \item Endereço completo
        \item Telefone e email
        \item Informações bancárias
        \item Condições de pagamento
    \end{itemize}
    \item Salve o cadastro
\end{enumerate}

\subsubsection{Serviços Utilizados}
\begin{itemize}
    \item \texttt{suppliersService.ts}: Gestão de fornecedores
    \item \texttt{purchaseService.ts}: Histórico de compras
\end{itemize}

\subsection{Rota /menu - Gestão de Cardápio}

\subsubsection{Funcionalidades}
\begin{itemize}
    \item Criação e gestão de pratos
    \item Definição de ingredientes e quantidades
    \item Cálculo automático de custos
    \item Planejamento de cardápio diário/semanal
    \item Análise de rentabilidade por prato
    \item Controle de sazonalidade
\end{itemize}

\subsubsection{Cadastro de Prato}
\begin{enumerate}
    \item Acesse "Novo Prato"
    \item Defina nome, categoria e preço
    \item Adicione ingredientes:
    \begin{itemize}
        \item Selecione produto do estoque
        \item Defina quantidade necessária
        \item Sistema calcula custo automaticamente
    \end{itemize}
    \item Configure informações nutricionais (opcional)
    \item Salve o prato
\end{enumerate}

\subsubsection{Planejamento de Cardápio}
\begin{enumerate}
    \item Selecione período (dia/semana/mês)
    \item Escolha pratos disponíveis
    \item Defina quantidades estimadas
    \item Sistema verifica disponibilidade de ingredientes
    \item Confirme o planejamento
\end{enumerate}

\subsubsection{Serviços Utilizados}
\begin{itemize}
    \item \texttt{menuService.ts}: Gestão de cardápio
    \item \texttt{recipeService.ts}: Receitas e ingredientes
    \item \texttt{planningService.ts}: Planejamento de produção
\end{itemize}

\subsection{Rota /reports - Relatórios e Análises}

\subsubsection{Funcionalidades Avançadas}
\begin{itemize}
    \item Relatórios financeiros detalhados
    \item Análises de vendas e margem
    \item Relatórios de movimentação
    \item Análise ABC de produtos
    \item Relatórios de fornecedores
    \item Análise preditiva com IA
    \item Exportação em múltiplos formatos (PDF, Excel, CSV)
\end{itemize}

\subsubsection{Tipos de Relatórios}

\paragraph{Relatórios Padrão}
\begin{itemize}
    \item \textbf{Estoque Atual}: Lista completa com valores
    \item \textbf{Movimentações}: Histórico de entradas/saídas
    \item \textbf{Produtos em Falta}: Itens com estoque zerado
    \item \textbf{Estoque Baixo}: Produtos abaixo do mínimo
    \item \textbf{Análise de Vendas}: Performance por período
\end{itemize}

\paragraph{Relatórios Avançados}
\begin{itemize}
    \item \textbf{Análise ABC}: Classificação por importância
    \item \textbf{Giro de Estoque}: Rotatividade de produtos
    \item \textbf{Margem de Contribuição}: Análise financeira
    \item \textbf{Sazonalidade}: Padrões de consumo
    \item \textbf{Previsão de Demanda}: IA preditiva
\end{itemize}

\subsubsection{Gráficos e Visualizações}
\begin{itemize}
    \item Gráficos de linha para tendências
    \item Gráficos de barras para comparações
    \item Gráficos de pizza para distribuições
    \item Mapas de calor para correlações
    \item Gráficos radar para análises multi-dimensionais
\end{itemize}

\subsubsection{Exportação de Dados}
\begin{enumerate}
    \item Selecione o relatório desejado
    \item Configure filtros e período
    \item Escolha formato de exportação:
    \begin{itemize}
        \item PDF com gráficos
        \item Excel com dados e tabelas dinâmicas
        \item CSV para análise externa
        \item PowerBI para dashboards
    \end{itemize}
    \item Clique em "Exportar"
\end{enumerate}

\subsubsection{Serviços Utilizados}
\begin{itemize}
    \item \texttt{reportsService.ts}: Geração de relatórios
    \item \texttt{advancedAnalyticsService.ts}: Análises estatísticas
    \item \texttt{advancedChartsService.ts}: Visualizações avançadas
    \item \texttt{advancedExportService.ts}: Exportação múltiplos formatos
    \item \texttt{aiAnalyticsService.ts}: Análises com IA
    \item \texttt{predictiveAnalyticsService.ts}: Análise preditiva
\end{itemize}

\subsection{Rota /ai - Inteligência Artificial}

\subsubsection{Funcionalidades de IA}
\begin{itemize}
    \item Análise preditiva de demanda
    \item Otimização de compras
    \item Detecção de anomalias
    \item Sugestões inteligentes de precificação
    \item Análise de padrões de consumo
    \item Insights automatizados
    \item Chatbot para consultas
\end{itemize}

\subsubsection{Análise Preditiva}
\begin{enumerate}
    \item Selecione produtos para análise
    \item Defina horizonte temporal
    \item Configure parâmetros:
    \begin{itemize}
        \item Sazonalidade
        \item Tendências históricas
        \item Fatores externos
    \end{itemize}
    \item Execute análise
    \item Visualize previsões e recomendações
\end{enumerate}

\subsubsection{Otimização de Estoque}
O sistema analisa histórico e sugere:
\begin{itemize}
    \item Pontos ótimos de reposição
    \item Quantidades econômicas de compra
    \item Produtos com potencial de descontinuação
    \item Oportunidades de cross-selling
\end{itemize}

\subsubsection{Serviços Utilizados}
\begin{itemize}
    \item \texttt{aiService.ts}: Integração com APIs de IA
    \item \texttt{aiAnalyticsService.ts}: Análises inteligentes
    \item \texttt{predictiveAnalyticsService.ts}: Modelos preditivos
\end{itemize}

\subsection{Rota /logs - Auditoria e Logs}

\subsubsection{Funcionalidades de Auditoria}
\begin{itemize}
    \item Rastreamento completo de ações
    \item Logs de acesso e segurança
    \item Histórico de alterações
    \item Análise de performance
    \item Monitoramento de erros
    \item Relatórios de conformidade
\end{itemize}

\subsubsection{Tipos de Log}
\begin{itemize}
    \item \textbf{Sistema}: Inicialização, erros, performance
    \item \textbf{Usuário}: Login, logout, ações realizadas
    \item \textbf{Estoque}: Movimentações, alterações
    \item \textbf{Segurança}: Tentativas de acesso, falhas
    \item \textbf{API}: Requisições, respostas, erros
\end{itemize}

\subsubsection{Filtros de Log}
\begin{itemize}
    \item Por usuário
    \item Por período
    \item Por tipo de ação
    \item Por nível de severidade
    \item Por módulo do sistema
\end{itemize}

\subsubsection{Serviços Utilizados}
\begin{itemize}
    \item \texttt{logsService.ts}: Gestão de logs
    \item \texttt{auditService.ts}: Trilha de auditoria
\end{itemize}

\subsection{Rota /settings - Configurações}

\subsubsection{Configurações Disponíveis}
\begin{itemize}
    \item Parâmetros gerais do sistema
    \item Configurações de estoque (pontos de reposição)
    \item Configurações financeiras (moedas, impostos)
    \item Configurações de notificação
    \item Integrações externas
    \item Backup e restauração
\end{itemize}

\subsubsection{Categorias de Configuração}

\paragraph{Sistema Geral}
\begin{itemize}
    \item Nome da empresa
    \item Logo e identidade visual
    \item Fuso horário
    \item Idioma padrão
    \item Formato de data/hora
\end{itemize}

\paragraph{Estoque}
\begin{itemize}
    \item Política de estoque mínimo global
    \item Configurações de alerta
    \item Métodos de valoração (FIFO, LIFO, Médio)
    \item Categorias padrão
\end{itemize}

\paragraph{Financeiro}
\begin{itemize}
    \item Moeda padrão
    \item Configurações de impostos
    \item Margem de lucro padrão
    \item Formas de pagamento
\end{itemize}

\subsubsection{Serviços Utilizados}
\begin{itemize}
    \item \texttt{settingsService.ts}: Gestão de configurações
    \item \texttt{configService.ts}: Parâmetros do sistema
\end{itemize}

\subsection{Rota /profile - Perfil do Usuário}

\subsubsection{Funcionalidades}
\begin{itemize}
    \item Edição de dados pessoais
    \item Alteração de senha
    \item Configurações de notificação
    \item Histórico de atividades
    \item Preferências de interface
    \item Configurações de segurança (2FA)
\end{itemize}

\subsubsection{Configurações de Perfil}
\begin{itemize}
    \item Nome completo
    \item Email de contato
    \item Telefone
    \item Foto de perfil
    \item Cargo/função
    \item Departamento
\end{itemize}

\subsubsection{Segurança}
\begin{itemize}
    \item Alteração de senha
    \item Autenticação de dois fatores
    \item Sessões ativas
    \item Dispositivos autorizados
\end{itemize}

\subsubsection{Serviços Utilizados}
\begin{itemize}
    \item \texttt{profileService.ts}: Gestão de perfil
    \item \texttt{authService.ts}: Segurança e autenticação
\end{itemize}

% ========================================
% SEÇÃO 6: SERVIÇOS E INTEGRAÇÃO
% ========================================

\section{Arquitetura de Serviços}

\subsection{Camada de Serviços}

O sistema implementa uma arquitetura em camadas com serviços especializados para cada domínio de negócio, garantindo separação de responsabilidades e facilidade de manutenção.

\subsubsection{Estrutura de Serviços}

\begin{longtable}{|p{4cm}|p{10cm}|}
\hline
\textbf{Serviço} & \textbf{Responsabilidades} \\
\hline
\endhead

\texttt{authService} & Autenticação, autorização, gestão de sessões \\
\hline

\texttt{productService} & CRUD de produtos, controle de estoque, categorias \\
\hline

\texttt{suppliersService} & Gestão de fornecedores, contratos, avaliações \\
\hline

\texttt{menuService} & Cardápio, receitas, ingredientes, planejamento \\
\hline

\texttt{salesService} & Vendas, faturamento, análise de performance \\
\hline

\texttt{reportsService} & Relatórios padrão, métricas, KPIs \\
\hline

\texttt{advancedAnalyticsService} & Análises estatísticas avançadas, correlações \\
\hline

\texttt{aiAnalyticsService} & Análises com inteligência artificial \\
\hline

\texttt{predictiveAnalyticsService} & Modelos preditivos, forecasting \\
\hline

\texttt{advancedChartsService} & Visualizações avançadas, gráficos customizados \\
\hline

\texttt{advancedExportService} & Exportação em múltiplos formatos \\
\hline

\texttt{logsService} & Auditoria, logs, rastreamento \\
\hline

\texttt{settingsService} & Configurações do sistema \\
\hline

\texttt{profileService} & Perfil do usuário, preferências \\
\hline

\end{longtable}

\subsection{Padrões de Implementação}

\subsubsection{Service Layer Pattern}

Cada serviço segue o padrão Service Layer, implementando:

\begin{lstlisting}[language=typescript, caption=Estrutura Base de Serviço]
export class BaseService {
  protected supabase = supabase

  // Operações CRUD padrão
  async getAll(): Promise<T[]>
  async getById(id: string): Promise<T>
  async create(data: Partial<T>): Promise<T>
  async update(id: string, data: Partial<T>): Promise<T>
  async delete(id: string): Promise<void>

  // Operações específicas do domínio
  // ...
}
\end{lstlisting}

\subsubsection{Error Handling}

Sistema padronizado de tratamento de erros:

\begin{lstlisting}[language=typescript, caption=Tratamento de Erros]
export interface ServiceError {
  code: string
  message: string
  details?: any
  timestamp: Date
}

export class ServiceException extends Error {
  constructor(
    public code: string,
    public message: string,
    public details?: any
  ) {
    super(message)
  }
}
\end{lstlisting}

\subsection{Integração com Supabase}

\subsubsection{Cliente Supabase}

\begin{lstlisting}[language=typescript, caption=Configuração Supabase]
import { createClient } from '@supabase/supabase-js'

const supabaseUrl = import.meta.env.VITE_SUPABASE_URL
const supabaseAnonKey = import.meta.env.VITE_SUPABASE_ANON_KEY

export const supabase = createClient(supabaseUrl, supabaseAnonKey, {
  auth: {
    autoRefreshToken: true,
    persistSession: true,
    detectSessionInUrl: false
  }
})
\end{lstlisting}

\subsubsection{Operações de Banco de Dados}

\begin{lstlisting}[language=typescript, caption=Exemplo de Operações CRUD]
export class ProductService {
  async getAllProducts(): Promise<Product[]> {
    const { data, error } = await supabase
      .from('produtos')
      .select(`
        *,
        categoria:categorias(*),
        fornecedor:suppliers(*)
      `)
      .order('nome')

    if (error) throw new ServiceException('DB_ERROR', error.message)
    return data
  }

  async createProduct(product: Partial<Product>): Promise<Product> {
    const { data, error } = await supabase
      .from('produtos')
      .insert(product)
      .select()
      .single()

    if (error) throw new ServiceException('CREATE_ERROR', error.message)
    return data
  }
}
\end{lstlisting}

% ========================================
% SEÇÃO 7: FUNCIONALIDADES AVANÇADAS
% ========================================

\section{Funcionalidades Avançadas}

\subsection{Sistema de Analytics Avançado}

\subsubsection{Análises Estatísticas}

O sistema implementa análises estatísticas robustas através do \texttt{advancedAnalyticsService}:

\begin{itemize}
    \item \textbf{Estatísticas Descritivas}: Média, mediana, moda, desvio padrão
    \item \textbf{Análise de Tendências}: Regressão linear, correlações
    \item \textbf{Detecção de Sazonalidade}: Padrões cíclicos, autocorrelação
    \item \textbf{Forecasting}: Previsão usando suavização exponencial
    \item \textbf{Detecção de Anomalias}: Baseada em Z-score e desvios
    \item \textbf{Análise de Distribuição}: Histogramas, bondade de ajuste
\end{itemize}

\subsubsection{Business Intelligence}

Funcionalidades de BI integradas:

\begin{lstlisting}[language=typescript, caption=KPIs Automatizados]
interface BusinessIntelligence {
  kpis: {
    revenue: {
      current: number
      growth: number
      target: number
      achievement: number
    }
    efficiency: {
      inventoryTurnover: number
      stockoutRate: number
      fillRate: number
      cycleTime: number
    }
    quality: {
      accuracyRate: number
      errorRate: number
      customerSatisfaction: number
    }
  }
  benchmarks: {
    industry: IndustryBenchmark
    internal: InternalBenchmark
  }
  recommendations: Recommendation[]
}
\end{lstlisting}

\subsection{Inteligência Artificial}

\subsubsection{Análise Preditiva}

O \texttt{predictiveAnalyticsService} implementa modelos preditivos:

\begin{itemize}
    \item \textbf{Previsão de Demanda}: Modelos ARIMA e suavização exponencial
    \item \textbf{Otimização de Estoque}: Cálculo de pontos de reposição ótimos
    \item \textbf{Análise de Churn}: Identificação de produtos em declínio
    \item \textbf{Segmentação ABC}: Classificação automática por importância
    \item \textbf{Análise de Cestas}: Market basket analysis
\end{itemize}

\subsubsection{Integração com APIs de IA}

\begin{lstlisting}[language=typescript, caption=Integração Gemini AI]
export class AIAnalyticsService {
  private readonly GEMINI_API_KEY = import.meta.env.VITE_GEMINI_API_KEY

  async generateInsights(data: any[]): Promise<string[]> {
    const prompt = this.buildAnalysisPrompt(data)

    const response = await this.callGeminiAPI(prompt)
    return this.parseInsights(response)
  }

  private async callGeminiAPI(prompt: string) {
    const response = await fetch(
      `https://generativelanguage.googleapis.com/v1beta/models/gemini-pro:generateContent?key=${this.GEMINI_API_KEY}`,
      {
        method: 'POST',
        headers: { 'Content-Type': 'application/json' },
        body: JSON.stringify({
          contents: [{ parts: [{ text: prompt }] }]
        })
      }
    )

    return response.json()
  }
}
\end{lstlisting}

\subsection{Sistema de Relatórios Avançados}

\subsubsection{Exportação Multi-Formato}

O \texttt{advancedExportService} suporta múltiplos formatos:

\begin{itemize}
    \item \textbf{PDF}: Relatórios formatados com gráficos
    \item \textbf{Excel}: Planilhas com dados e tabelas dinâmicas
    \item \textbf{CSV}: Dados estruturados para análise externa
    \item \textbf{PowerBI}: Datasets otimizados para dashboards
    \item \textbf{Imagem}: Screenshots de dashboards
\end{itemize}

\subsubsection{Customização de Relatórios}

\begin{lstlisting}[language=typescript, caption=Opções de Exportação]
interface ExportOptions {
  format: 'pdf' | 'excel' | 'csv' | 'json' | 'powerbi' | 'image'
  includeAI: boolean
  includePredictive: boolean
  includeCharts: boolean
  includeRawData: boolean
  customSections?: string[]
  branding?: {
    logo?: string
    companyName?: string
    colors?: {
      primary: string
      secondary: string
    }
  }
}
\end{lstlisting}

\subsection{Visualizações Avançadas}

\subsubsection{Tipos de Gráfico Suportados}

O \texttt{advancedChartsService} implementa visualizações sofisticadas:

\begin{itemize}
    \item \textbf{Line Charts}: Tendências temporais
    \item \textbf{Bar/Column Charts}: Comparações categóricas
    \item \textbf{Pie/Doughnut Charts}: Distribuições percentuais
    \item \textbf{Radar Charts}: Análises multi-dimensionais
    \item \textbf{Scatter Plots}: Correlações entre variáveis
    \item \textbf{Bubble Charts}: Três dimensões de dados
    \item \textbf{Heatmaps}: Matrizes de correlação
    \item \textbf{Gauge Charts}: Indicadores de performance
\end{itemize}

\subsubsection{Configuração de Gráficos}

\begin{lstlisting}[language=typescript, caption=Configuração de Gráfico]
interface AdvancedChartData {
  type: 'line' | 'bar' | 'doughnut' | 'radar' | 'bubble' | 'scatter' | 'polarArea' | 'heatmap'
  data: ChartData
  options: ChartOptions
  config?: ChartConfiguration
}

export class AdvancedChartsService {
  generateSalesPerformanceChart(
    salesData: any[],
    period: string
  ): AdvancedChartData {
    const labels = salesData.map(item => item.date)
    const sales = salesData.map(item => item.total)
    const movingAverage = this.calculateMovingAverage(sales, 7)

    return {
      type: 'line',
      data: {
        labels,
        datasets: [
          {
            label: 'Vendas Diárias',
            data: sales,
            borderColor: '#667eea',
            backgroundColor: 'rgba(102, 126, 234, 0.1)',
            fill: true,
            tension: 0.4
          },
          {
            label: 'Média Móvel (7 dias)',
            data: movingAverage,
            borderColor: '#f093fb',
            backgroundColor: 'transparent',
            borderWidth: 2,
            borderDash: [5, 5],
            pointRadius: 0
          }
        ]
      },
      options: this.getAdvancedLineOptions('Vendas por Período', 'R$')
    }
  }
}
\end{lstlisting}

% ========================================
% SEÇÃO 8: SEGURANÇA E PERFORMANCE
% ========================================

\section{Segurança e Performance}

\subsection{Medidas de Segurança}

\subsubsection{Autenticação e Autorização}

\begin{itemize}
    \item \textbf{Supabase Auth}: Sistema robusto de autenticação
    \item \textbf{JWT Tokens}: Tokens seguros com expiração automática
    \item \textbf{Row Level Security}: Controle granular de acesso
    \item \textbf{HTTPS Only}: Comunicação criptografada
    \item \textbf{Session Management}: Controle de sessões ativas
\end{itemize}

\subsubsection{Proteção de Dados}

\begin{itemize}
    \item \textbf{Criptografia}: Senhas hasheadas com bcrypt
    \item \textbf{Sanitização}: Prevenção de SQL injection
    \item \textbf{Validação}: Validação rigorosa de inputs
    \item \textbf{Auditoria}: Log completo de ações
    \item \textbf{Backup}: Backup automático de dados
\end{itemize}

\subsubsection{Políticas RLS}

\begin{lstlisting}[language=sql, caption=Políticas de Segurança]
-- Política para acesso a produtos
CREATE POLICY "authenticated_users_products" ON produtos
    FOR ALL USING (auth.role() = 'authenticated');

-- Política para logs (somente leitura)
CREATE POLICY "read_logs" ON logs
    FOR SELECT USING (auth.role() = 'authenticated');

-- Política para configurações (somente admin)
CREATE POLICY "admin_settings" ON app_settings
    FOR ALL USING (auth.jwt() ->> 'role' = 'admin');
\end{lstlisting}

\subsection{Otimizações de Performance}

\subsubsection{Frontend}

\begin{itemize}
    \item \textbf{Code Splitting}: Carregamento sob demanda
    \item \textbf{Lazy Loading}: Componentes carregados quando necessário
    \item \textbf{Virtual Scrolling}: Listas grandes otimizadas
    \item \textbf{Caching}: Cache inteligente de dados
    \item \textbf{Compression}: Assets comprimidos
\end{itemize}

\subsubsection{Backend/Database}

\begin{itemize}
    \item \textbf{Indexação}: Índices otimizados para consultas
    \item \textbf{Connection Pooling}: Pool de conexões eficiente
    \item \textbf{Query Optimization}: Consultas SQL otimizadas
    \item \textbf{Caching Strategy}: Cache de resultados frequentes
    \item \textbf{Real-time Subscriptions}: Atualizações em tempo real
\end{itemize}

\subsubsection{Monitoramento}

\begin{lstlisting}[language=typescript, caption=Monitoramento de Performance]
export class PerformanceMonitor {
  static measureExecutionTime<T>(
    fn: () => Promise<T>,
    operation: string
  ): Promise<T> {
    const start = performance.now()

    return fn().finally(() => {
      const end = performance.now()
      const duration = end - start

      console.log(`${operation} took ${duration.toFixed(2)}ms`)

      // Log performance metrics
      this.logMetric({
        operation,
        duration,
        timestamp: new Date()
      })
    })
  }
}
\end{lstlisting}

% ========================================
% SEÇÃO 9: DEPLOYMENT E MANUTENÇÃO
% ========================================

\section{Deployment e Manutenção}

\subsection{Processo de Build}

\subsubsection{Scripts de Build}

\begin{lstlisting}[language=json, caption=Package.json Scripts]
{
  "scripts": {
    "dev": "vite",
    "build": "vue-tsc && vite build",
    "preview": "vite preview",
    "lint": "eslint src --ext .vue,.ts,.tsx",
    "type-check": "vue-tsc --noEmit"
  }
}
\end{lstlisting}

\subsubsection{Configuração de Build}

\begin{lstlisting}[language=typescript, caption=Vite Config]
import { defineConfig } from 'vite'
import vue from '@vitejs/plugin-vue'
import { resolve } from 'path'

export default defineConfig({
  plugins: [vue()],
  resolve: {
    alias: {
      '@': resolve(__dirname, 'src')
    }
  },
  build: {
    target: 'es2015',
    outDir: 'dist',
    assetsDir: 'assets',
    sourcemap: false,
    rollupOptions: {
      output: {
        manualChunks: {
          vendor: ['vue', 'vue-router', 'pinia'],
          charts: ['chart.js', 'vue-chartjs'],
          utils: ['axios', 'date-fns']
        }
      }
    }
  }
})
\end{lstlisting}

\subsection{Deployment em Produção}

\subsubsection{Vercel (Recomendado)}

\begin{enumerate}
    \item Configure as variáveis de ambiente no painel da Vercel
    \item Conecte o repositório GitHub
    \item A build será executada automaticamente
    \item Configure domínio personalizado (opcional)
\end{enumerate}

\begin{lstlisting}[language=json, caption=vercel.json]
{
  "builds": [
    {
      "src": "package.json",
      "use": "@vercel/static-build",
      "config": {
        "distDir": "dist"
      }
    }
  ],
  "routes": [
    {
      "src": "/(.*)",
      "dest": "/index.html"
    }
  ]
}
\end{lstlisting}

\subsubsection{Netlify}

\begin{lstlisting}[language=toml, caption=netlify.toml]
[build]
  command = "npm run build"
  publish = "dist"

[[redirects]]
  from = "/*"
  to = "/index.html"
  status = 200

[build.environment]
  NODE_VERSION = "18"
\end{lstlisting}

\subsection{Monitoramento e Logs}

\subsubsection{Monitoramento de Aplicação}

\begin{itemize}
    \item \textbf{Error Tracking}: Sentry ou similar
    \item \textbf{Performance Monitoring}: Web Vitals
    \item \textbf{Analytics}: Google Analytics 4
    \item \textbf{Uptime Monitoring}: Pingdom ou similar
\end{itemize}

\subsubsection{Logs Estruturados}

\begin{lstlisting}[language=typescript, caption=Sistema de Logs]
export enum LogLevel {
  DEBUG = 0,
  INFO = 1,
  WARN = 2,
  ERROR = 3
}

export interface LogEntry {
  timestamp: Date
  level: LogLevel
  message: string
  metadata?: any
  userId?: string
  sessionId?: string
}

export class Logger {
  static log(level: LogLevel, message: string, metadata?: any) {
    const entry: LogEntry = {
      timestamp: new Date(),
      level,
      message,
      metadata,
      userId: this.getCurrentUserId(),
      sessionId: this.getSessionId()
    }

    // Send to logging service
    this.sendToSupabase(entry)
  }
}
\end{lstlisting>

\subsection{Backup e Recuperação}

\subsubsection{Estratégia de Backup}

\begin{itemize}
    \item \textbf{Backup Diário}: Supabase automático
    \item \textbf{Backup Semanal}: Export completo de dados
    \item \textbf{Point-in-time Recovery}: Até 7 dias
    \item \textbf{Backup de Código}: Git com múltiplos remotes
\end{itemize}

\subsubsection{Procedimento de Recuperação}

\begin{enumerate}
    \item Identificar ponto de falha
    \item Selecionar backup apropriado
    \item Restaurar via Supabase Dashboard
    \item Verificar integridade dos dados
    \item Executar testes de validação
    \item Comunicar usuários sobre restauração
\end{enumerate}

% ========================================
% SEÇÃO 10: TROUBLESHOOTING
% ========================================

\section{Troubleshooting e FAQ}

\subsection{Problemas Comuns}

\subsubsection{Erros de Autenticação}

\begin{warningbox}{Erro: "Invalid credentials"}
\textbf{Causa}: Credenciais incorretas ou sessão expirada \\
\textbf{Solução}:
\begin{itemize}
    \item Verificar email/senha
    \item Limpar cache do navegador
    \item Verificar configurações Supabase
    \item Renovar tokens de sessão
\end{itemize}
\end{warningbox}

\subsubsection{Problemas de Performance}

\begin{warningbox}{Lentidão na Aplicação}
\textbf{Causas Possíveis}:
\begin{itemize}
    \item Consultas não otimizadas
    \item Cache desatualizado
    \item Muitos dados na memória
    \item Conexão lenta com BD
\end{itemize}
\textbf{Soluções}:
\begin{itemize}
    \item Otimizar queries SQL
    \item Implementar paginação
    \item Usar lazy loading
    \item Verificar índices do BD
\end{itemize}
\end{warningbox}

\subsubsection{Erros de Build}

\begin{warningbox}{TypeScript Compilation Errors}
\textbf{Comando de Diagnóstico}: \texttt{npm run type-check} \\
\textbf{Soluções Comuns}:
\begin{itemize}
    \item Verificar tipos importados
    \item Atualizar dependências
    \item Limpar cache: \texttt{rm -rf node\_modules package-lock.json}
    \item Reinstalar: \texttt{npm install}
\end{itemize}
\end{warningbox}

\subsection{Comandos de Diagnóstico}

\subsubsection{Verificação de Sistema}

\begin{lstlisting}[language=bash, caption=Scripts de Diagnóstico]
# Verificar versões
node --version
npm --version

# Verificar dependências
npm ls

# Verificar tipos TypeScript
npm run type-check

# Verificar lint
npm run lint

# Teste de build
npm run build

# Verificar configuração Supabase
npx supabase status
\end{lstlisting}

\subsubsection{Logs de Debug}

\begin{lstlisting}[language=typescript, caption=Debug Mode]
// Ativar modo debug
localStorage.setItem('DEBUG_MODE', 'true')

// Debug de queries Supabase
const { data, error } = await supabase
  .from('produtos')
  .select('*')
  .explain({ analyze: true, verbose: true })
\end{lstlisting}

\subsection{FAQ - Perguntas Frequentes}

\subsubsection{Funcionalidades}

\begin{itemize}
    \item \textbf{P}: Como importar produtos em massa? \\
    \textbf{R}: Use a funcionalidade de importação Excel na rota /inventory

    \item \textbf{P}: É possível customizar relatórios? \\
    \textbf{R}: Sim, na rota /reports há opções avançadas de customização

    \item \textbf{P}: Como configurar alertas de estoque baixo? \\
    \textbf{R}: Configure limites mínimos em /settings e ative notificações
\end{itemize}

\subsubsection{Técnicas}

\begin{itemize}
    \item \textbf{P}: Como fazer backup dos dados? \\
    \textbf{R}: Use o painel Supabase ou a exportação automática em /reports

    \item \textbf{P}: É possível integrar com outros sistemas? \\
    \textbf{R}: Sim, através da API REST do Supabase ou webhooks customizados

    \item \textbf{P}: Como otimizar performance para muitos produtos? \\
    \textbf{R}: Use filtros, paginação e índices adequados no banco
\end{itemize}

% ========================================
% SEÇÃO 11: API e INTEGRAÇÕES
% ========================================

\section{API e Integrações}

\subsection{API Supabase}

\subsubsection{Endpoints Principais}

O Supabase gera automaticamente uma API REST para todas as tabelas:

\begin{lstlisting}[language=bash, caption=Exemplos de Endpoints]
# Produtos
GET    /rest/v1/produtos
POST   /rest/v1/produtos
PATCH  /rest/v1/produtos?id=eq.{id}
DELETE /rest/v1/produtos?id=eq.{id}

# Movimentações
GET    /rest/v1/movements
POST   /rest/v1/movements

# Relatórios
GET    /rest/v1/reports
POST   /rest/v1/reports
\end{lstlisting}

\subsubsection{Autenticação API}

\begin{lstlisting}[language=bash, caption=Headers de Autenticação]
curl -X GET 'https://seu-projeto.supabase.co/rest/v1/produtos' \
  -H "apikey: SUA_ANON_KEY" \
  -H "Authorization: Bearer SEU_JWT_TOKEN"
\end{lstlisting}

\subsection{Integrações Externas}

\subsubsection{Google Gemini AI}

\begin{lstlisting}[language=typescript, caption=Integração Gemini]
export class AIService {
  private readonly apiKey = import.meta.env.VITE_GEMINI_API_KEY
  private readonly baseUrl = 'https://generativelanguage.googleapis.com/v1beta'

  async generateContent(prompt: string): Promise<string> {
    const response = await fetch(
      `${this.baseUrl}/models/gemini-pro:generateContent?key=${this.apiKey}`,
      {
        method: 'POST',
        headers: { 'Content-Type': 'application/json' },
        body: JSON.stringify({
          contents: [{ parts: [{ text: prompt }] }]
        })
      }
    )

    const data = await response.json()
    return data.candidates[0].content.parts[0].text
  }
}
\end{lstlisting}

\subsubsection{Webhooks}

Configurar webhooks para eventos importantes:

\begin{lstlisting}[language=sql, caption=Webhook para Estoque Baixo]
-- Função para webhook
CREATE OR REPLACE FUNCTION notify_low_stock()
RETURNS TRIGGER AS $$
BEGIN
  IF NEW.current_stock <= NEW.min_stock THEN
    PERFORM net.http_post(
      url := 'https://seu-webhook-url.com/low-stock',
      headers := '{"Content-Type": "application/json"}'::jsonb,
      body := json_build_object(
        'product_id', NEW.id,
        'product_name', NEW.nome,
        'current_stock', NEW.current_stock,
        'min_stock', NEW.min_stock
      )::jsonb
    );
  END IF;

  RETURN NEW;
END;
$$ LANGUAGE plpgsql;

-- Trigger
CREATE TRIGGER low_stock_webhook
  AFTER UPDATE OF current_stock ON produtos
  FOR EACH ROW
  EXECUTE FUNCTION notify_low_stock();
\end{lstlisting}

% ========================================
% SEÇÃO 12: CONCLUSÃO
% ========================================

\section{Conclusão}

\subsection{Resumo das Capacidades}

O \textbf{GestãoZe System} representa uma solução completa e moderna para gestão de estoque, combinando:

\begin{successbox}{Pontos Fortes do Sistema}
\begin{itemize}
    \item \textbf{Arquitetura Moderna}: Vue.js 3 com TypeScript
    \item \textbf{Backend Robusto}: Supabase com PostgreSQL
    \item \textbf{Análises Avançadas}: IA e análise preditiva
    \item \textbf{Interface Intuitiva}: UX otimizada para produtividade
    \item \textbf{Segurança Robusta}: RLS e autenticação segura
    \item \textbf{Escalabilidade}: Arquitetura preparada para crescimento
    \item \textbf{Relatórios Ricos}: Múltiplos formatos de exportação
    \item \textbf{Integração Flexível}: APIs e webhooks
\end{itemize}
\end{successbox}

\subsection{Roadmap de Desenvolvimento}

\subsubsection{Próximas Funcionalidades}
\begin{itemize}
    \item \textbf{Mobile App}: Aplicativo nativo com React Native
    \item \textbf{E-commerce Integration}: Integração com plataformas de venda
    \item \textbf{Advanced AI}: Modelos de deep learning customizados
    \item \textbf{Multi-tenancy}: Suporte para múltiplas empresas
    \item \textbf{IoT Integration}: Sensores de estoque automáticos
    \item \textbf{Blockchain}: Rastreabilidade de produtos
\end{itemize}

\subsection{Suporte e Comunidade}

\subsubsection{Recursos de Apoio}
\begin{itemize}
    \item \textbf{Documentação}: Sempre atualizada
    \item \textbf{Issues GitHub}: Relatório de bugs e sugestões
    \item \textbf{Comunidade}: Fórum de discussão
    \item \textbf{Tutoriais}: Vídeos e guias passo a passo
    \item \textbf{API Reference}: Documentação completa da API
\end{itemize}

\subsubsection{Contribuição}
O sistema é desenvolvido seguindo boas práticas de código aberto:
\begin{itemize}
    \item \textbf{Code Reviews}: Revisão rigorosa de código
    \item \textbf{Testing}: Cobertura de testes abrangente
    \item \textbf{CI/CD}: Pipeline automatizado de deployment
    \item \textbf{Documentation}: Documentação viva e atualizada
\end{itemize}

\vspace{2cm}

\begin{center}
\rule{10cm}{0.4pt} \\
\vspace{0.5cm}
\textbf{\large{GestãoZe System v1.0.0}} \\
\textit{Sistema de Gestão de Estoque Inteligente} \\
\vspace{0.2cm}
Desenvolvido com Vue.js 3, TypeScript e Supabase \\
\textcolor{primary}{\href{https://gestao.restpedacinhodoceu.com.br}{gestao.restpedacinhodoceu.com.br}}
\end{center}

\end{document}